%%\chapter{GWT}
\chapter[GWT]{GWT\cite{GWT}}
\glossary{name={GWT}, description={Google Web Toolkit}}
\setcounter{secnumdepth}{5}
\setcounter{tocdepth}{5}

\section{Panoramica}
	  \subsection{Che cos'è}
	
	  
	GWT è uno strumento di sviluppo per la costruzione e l'ottimizzazione di applicazioni browser-based complesse.
	il suo scopo è quello di alzare la produttività e le prestazioni delle applicazioni senza che lo sviluppatore
	si debba preoccupare dei problemi legati al cross-browsing quindi propri del linguaggio javascript e delle XMLHttpRequest.
	GWT e usato in molti prodotti Google come AdWords,AdSense,Flights (far vedere codice applicazione),Hotel Finder, Offers, Wallet, Blogger.
	GWT è open source ed è distribuito sotto licenza Apache 2.0.
	I punti di forza di GWT sono la riusabilità del codice , la possibilità di 
	creare pagine web dinamiche mediante chiamate asincrone AJAX, il bookmarking, la localizzazione e la portabilità fra differenti browser
	  \subsection{Componenti software}
	  \subsection{Sviluppare in GWT}
		\subsubsection{Scrivere}
		
		\subsubsection{Debuggare}

		\subsubsection{Ottimizzare}
		
		\subsubsection{Eseguire}
		
\section{Installare GWT}
	  \subsection{Installazione GWT SDK}
	  \subsection{Configurare IDE}
\section{Organizzazione del progetto}
\section{Compilazione e Debugging}
\section{Client-side}
\section{Server-side}
\section{Compatibilità con linguaggio}
\section{User interface}
\section{Rilascio Progetto}
\chapter{Framekork di Sviluppo Web}
\setcounter{secnumdepth}{5}
\setcounter{tocdepth}{5}

\section{Definizione}
Un framework in informatica e specificatamente nello sviluppo software, 
è un'architettura (o più impropriamente struttura) logica di supporto
(spesso un'implementazione logica di un particolare design pattern) su 
cui un software può essere progettato e realizzato, spesso facilitandone 
lo sviluppo da parte del programmatore.

Un framework è un'astrazione software la quale realizzazione fornisce funzionalità 
generiche che possono essere modificate o aggiunte selettivamente dal codice scritto 
dall'utente fornendo in tal modo uno specifico compito all'applicazione.Un framework
facilita lo sviluppo di applicazioni in quanto fornisce una base di 
componenti riutillizabili in diversi progetti senza dover riscrivere queste componenti
comuni ma mettendole a disposizione del programmatore.
Il fine ultimo di un framework è appunto quello di costruire una base di partenza nello 
sviluppo, come le fondamenta su cui poi si potranno costruire edifici diversissimi.

\section{Caratteristiche}
	  \subsection{Architettura Logica}
	  \subsection{Blocchi per funzioni ripetitive}
	  \subsection{Processi di Sviluppo}
		      \subsubsection{Indietro}
		      \subsubsection{Avanti}
	  \subsection{Catalogo Framekork}
	  \subsection{Confronto}
\section{Come Scegliere}
\section{GWT e mercato}
 
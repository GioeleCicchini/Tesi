\documentclass[hidelinks,a4paper,12pt]{tesiinfo}

%%Pacchetti utili anche se non necessari
\usepackage{amsfonts}

\usepackage{amsmath}

\usepackage{latexsym}

\usepackage{tabularx}
%% Supporto alla lingua italiana, dipende dal vostro ambiente latex
\usepackage[italian]{babel}
%% Supporto per i caratteri accentati
\usepackage[utf8]{inputenc}
%% Supporto alle figure
\usepackage{graphicx}
%% Supporto alle figure wrappate nel testo
 \usepackage{wrapfig}
%% Supporto alle didascalie
\usepackage{caption}
%% Supporto alle didascalie per figure composte
\usepackage{subcaption}
%% Supporto ai link esterni (URL)
\usepackage[bookmarks=true]{hyperref}

%% Supporto alla creazione di colonne multiple nel documento
\usepackage{multicol}
%% Supporto al comando \cite per BibTex
\usepackage{cite}
%% Supporto per i listati di codice
\usepackage{listings}
%% Supporto allo stile per il codice PHP creato da Nicola Sacco e Daniele Di Pompeo
\input{progrLang.sty}
%% Supporto al glossario
\usepackage[style=altlist, toc=true]{glossary} % can be obtained from http://www.ctan.org/tex-archive/macros/latex/contrib/glossary/
%% Richiesta di creazione del glossario
\makeglossary
%% Inclusione del file con i termini del glossario



\titolo{GWT}
\laureando{Gioele Cicchini 221034}
\relatore{Serafino Cicerone}
\annoaccademico{2015/2016}

%%Usare i seguenti comandi se si ha un correlatore:
%\setcorrelatoreuno
%\correlatoreuno{Nome e cognome del correlatore}

%%Usare i seguenti comandi se si hanno due correlatori (NB: questi comandi sono alternativi a quelli precendenti):
%\setcorrelatoredue
%\correlatoreuno{Nome e cognome del primo correlatore}
%\correlatoredue{Nome e cognome del secondo correlatore}

%%Usare i seguenti comandi se si ha un relatore esterno (NB: questi comandi possono essere utilizzati con quelli precedenti):
%\setesterno
%\relatoreesterno{Titolo, nome e cognome del relatore esterno}

%%Usare i seguenti comandi se si sta scrivendo una tesi di laurea specialistica
%\setspecialistica

\begin{document}
\storeglosentry{Esempio}{name={Nome da visualizzare}, description={Descrizione nel glossario}}

\maketitle
   
  \begin{dedication}
  \textit{Dedica a piè pagina}
  \end{dedication}



\contentspage
  
  \chapter{Introduzione}
Capitolo introduttivo di prova

  \chapter{Framekork di Sviluppo Web}
\setcounter{secnumdepth}{5}
\setcounter{tocdepth}{5}

\section{Definizione}
Un framework in informatica e specificatamente nello sviluppo software, 
è un'architettura (o più impropriamente struttura) logica di supporto
(spesso un'implementazione logica di un particolare design pattern) su 
cui un software può essere progettato e realizzato, spesso facilitandone 
lo sviluppo da parte del programmatore.

Un framework è un'astrazione software la quale realizzazione fornisce funzionalità 
generiche che possono essere modificate o aggiunte selettivamente dal codice scritto 
dall'utente fornendo in tal modo uno specifico compito all'applicazione.Un framework
facilita lo sviluppo di applicazioni in quanto fornisce una base di 
componenti riutillizabili in diversi progetti senza dover riscrivere queste componenti
comuni ma mettendole a disposizione del programmatore.
Il fine ultimo di un framework è appunto quello di costruire una base di partenza nello 
sviluppo, come le fondamenta su cui poi si potranno costruire edifici diversissimi.

\section{Caratteristiche}
	  \subsection{Architettura Logica}
	  \subsection{Blocchi per funzioni ripetitive}
	  \subsection{Processi di Sviluppo}
		      \subsubsection{Indietro}
		      \subsubsection{Avanti}
	  \subsection{Catalogo Framekork}
	  \subsection{Confronto}
\section{Come Scegliere}
\section{GWT e mercato}
 
   %%\chapter{GWT}
\chapter[GWT]{GWT\cite{GWT}}
\glossary{name={GWT}, description={Google Web Toolkit}}
\setcounter{secnumdepth}{5}
\setcounter{tocdepth}{5}

\section{Panoramica}
	  \subsection{Che cos'è}
	
	  
	GWT è uno strumento di sviluppo per la costruzione e l'ottimizzazione di applicazioni browser-based complesse.
	il suo scopo è quello di alzare la produttività e le prestazioni delle applicazioni senza che lo sviluppatore
	si debba preoccupare dei problemi legati al cross-browsing quindi propri del linguaggio javascript e delle XMLHttpRequest.
	GWT e usato in molti prodotti Google come AdWords,AdSense,Flights (far vedere codice applicazione),Hotel Finder, Offers, Wallet, Blogger.
	GWT è open source ed è distribuito sotto licenza Apache 2.0.
	I punti di forza di GWT sono la riusabilità del codice , la possibilità di 
	creare pagine web dinamiche mediante chiamate asincrone AJAX, il bookmarking, la localizzazione e la portabilità fra differenti browser
	  \subsection{Componenti software}
	  \subsection{Sviluppare in GWT}
		\subsubsection{Scrivere}
		
		\subsubsection{Debuggare}

		\subsubsection{Ottimizzare}
		
		\subsubsection{Eseguire}
		
\section{Installare GWT}
	  \subsection{Installazione GWT SDK}
	  \subsection{Configurare IDE}
\section{Organizzazione del progetto}
\section{Compilazione e Debugging}
\section{Client-side}
\section{Server-side}
\section{Compatibilità con linguaggio}
\section{User interface}
\section{Rilascio Progetto}
 \chapter{Progetto}
\setcounter{secnumdepth}{5}
\setcounter{tocdepth}{5}

\section{Visione}
\section{Progettazione}
\section{Realizzazione}
  \chapter{Conclusione}
Conclusione
\begin{itemize}
 \item item di prova 1
 \item item di prova 2
\end{itemize}

  

\appendice
 
 
 \bibliography{bibliografia/bibliografia}{}
 \bibliographystyle{tufte}
 
 \input{appendici/appendice.tex} 
 \printglossary
\begin{dedication}
  
\end{dedication}
\begin{dedication}
  \textit{Dedica a fine pagina}
\end{dedication}

\end{document}

%%%%%%%%%%%%%%%%%%%%%%%%%%%%%%%%%%%%%%%%%%%%%%%%%%%%%%%%%%%%%%%%%%%%%%%%%%%%%%%%

